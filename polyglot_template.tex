% !TEX program = lualatex
\documentclass{article}

%% ---- Set up paper margin ---- %%
%\usepackage[a4paper,top=1in,bottom=1in,left=1in,right=1in]{geometry}

% use multiple languages
%% ---- ตั้งค่าให้ตัดคำภาษาไทย ---- %%
\XeTeXlinebreaklocale "th"
\XeTeXlinebreakskip = 0pt plus 0pt % เพิ่มความกว้างเว้นวรรคให้ความยาวแต่ละบรรทัดเท่ากัน

%% ---- font settings ---- %%
\usepackage{fontspec}
% \usepackage{xltxtra}
% \usepackage{fonts-tlwg}
\defaultfontfeatures{Mapping=tex-text} % map LaTeX formating, e.g., ``'', to match the current font
% To change the main font, uncomment one of the below command.
% \setmainfont{TeX Gyre Termes} % Free Times
% \setsansfont{TeX Gyre Heros} % Free Helvetica
% \setmonofont{TeX Gyre Cursor} % Free Courier
\newfontfamily{\thaifont}[%
	ItalicFont={Laksaman-Italic.otf},%
	BoldFont={Laksaman-Bold.otf},%
	BoldItalicFont={Laksaman-BoldItalic.otf},%
	Script=Thai,%
	Scale=MatchLowercase,%
	WordSpace=1.25,%
	Mapping=tex-text,%
]{Laksaman.otf} % ตั้งฟอนต์หลักภาษาไทย
\newenvironment{thailang}{\thaifont}{} % create environment for Thai language
\usepackage[Latin,Thai]{ucharclasses} % ตั้งค่าให้ใช้ "thailang" environment เฉพาะ string ที่เป็น Unicode ภาษาไทย
\setTransitionTo{Thai}{\begin{thailang}}
		\setTransitionFrom{Thai}{\end{thailang}}
\newfontfamily\thaifontsf{IBM Plex Sans Thai}
\newfontfamily\thaifonttt{TlwgTypist.otf}
\newcommand{\thaisf}[1]{{\uccoff\thaifontsf #1\uccon}}
\newcommand{\thaitt}[1]{{\uccoff\thaifonttt #1\uccon}}

%% ---- allow CJK usage ---- %%
\usepackage{xeCJK} % this should be called before Polyglossia
% \setCJKmainfont[
%     Path = {\string~/.local/share/fonts/},
%     Extension = .ttf,
%     % AutoFakeBold=true,
%     % AutoFakeSlant=true
%     ]{NotoSerifCJKtc-VF}
% \setCJKsansfont[
%     Path = {\string/usr/share/fonts/google-noto-sans-cjk-vf-fonts/},
%     Extension = .ttf
%     ]{NotoSansCJK-VF}
% \setCJKmonofont[
%     Path = {\string/usr/share/fonts/google-noto-sans-mono-cjk-vf-fonts/},
%     Extension = .ttf
%     ]{NotoSansMonoCJK-VF}
% \setCJKmainfont{Source Han Serif JP}
\setCJKmainfont[ ItalicFont=AR PL UKai TW ]{Source Han Serif JP}
\setCJKsansfont{Source Han Sans JP}
\setCJKmonofont{Source Han Mono}
%%% Define fonts for Japanese and Korean
% \newCJKfontfamily\japanesefont{Noto Serif CJK JP}
% \setCJKmainfont{Noto Serif CJK TC}
% \setCJKsansfont{Noto Sans CJK TC}
% \setCJKmonofont{Noto Sans Mono CJK TC}
% \setCJKmainfont{Noto Serif CJK JP}
% \setCJKsansfont{Noto Sans CJK JP}
% \setCJKmonofont{Noto Sans Mono CJK JP}
% \setCJKmainfont{IPAMincho}
% \setCJKsansfont{IPAGothic}
% \setCJKmonofont{IPAGothic}

%% ---- spacing between lines ---- %%
\usepackage{setspace}
% \singlespacing % default setting
% \onehalfspacing % recommend using this for Thai language

%% ---- using alphabatic language ---- %%
\usepackage{polyglossia}
\setdefaultlanguage{english} % it is preferrable to set English as the main language, since the numeric system is compatible with most LaTeX features such as 'enumerate' and so on
\setotherlanguages{thai}

\AtBeginDocument\captionsthai % allow captions to be in Thai


%% ---- math packages ---- %%
\usepackage{amsmath}
\usepackage{amssymb}
\usepackage{bm} % same functionality as \mathbf{} but for greek letters
\usepackage{derivative} % macros for typesetting derivatives
\numberwithin{equation}{section} % equation numbers are formatted as <#Section>.<#eq in the section>

%% ---- define math environment ---- %%
\usepackage{amsthm}
\newtheorem{definition}{Definition}[section]
\newtheorem{proposition}[definition]{Proposition}
\newtheorem{theorem}[definition]{Theorem}
\newtheorem{corollary}{Corollary}[definition]
\newtheorem{remark}{Remark}[definition]
\newtheorem{lemma}[definition]{Lemma}

%% ---- hyperref settings ---- %%
\usepackage{hyperref}
\usepackage{url}
\usepackage{cite}
\usepackage{xcolor}
\hypersetup{
	colorlinks,
	linkcolor={red!50!black},
	citecolor={green!50!black},
	urlcolor={blue!80!black}
}

%% ---- misc. packages ---- %%
\usepackage{enumitem} % allow customizing list environments: enumerate, itemize and description.
\usepackage{mhchem} % use chemistry notation
\usepackage{lipsum}
\usepackage{metalogo} % for extended LaTeX logo such as XeTeX
\usepackage{subcaption} % allowing subfigure environment
% \usepackage[section]{placeins} % ensure floats do not go into the next section and allow the use of \FloatBarrier
\usepackage{graphicx} % allow cropping and rotating images

%% ---- title, authors, and dates ---- %%
\usepackage{authblk}
\title{Template for multilingual typesetting using \LuaTeX}
\author[1]{Aiden Sintavanuruk \\ (ชนกนันท์ สินธวานุรักษ์, 馬予棟)}
% \affil[1]{Department of Physiology, Faculty of Medicine Siriraj Hospital, Mahidol University}
% \author[2,3]{XXXX XXXX}
% \affil[2]{Department of XXXX, XXXX University}
% \affil[3]{Department of XXXX, XXXX University}
\date{\today}

\babelpatterns{ระ2หว่าง}

\begin{document}
% \sloppy % ช่วยตัดคำภาษาไทย
\begin{otherlanguage}{english}
	\maketitle
	\begin{abstract}
		\textsf{This} document use \texttt{babel} and \texttt{luatexja} packages to provide multilingual typesetting in Thai, Chinese and Japanese.
		ฟอนต์ภาษาไทย \texttt{Laksaman} เป็นของโครงการ TLWG (Thai Linux Working Group) ที่คล้ายกับ \texttt{TH SarabunPSK} \textsf{ที่ใช้สำหรับการจัดทำเอกสารใน MS Word}
	\end{abstract}
\end{otherlanguage}

\section{Alphabatic scripts}
Alphabatic scripts, e.g., Thai, Devanagari, Arabic, Cyrillic, etc., can be used simultaneously by using \texttt{babel} package.
To ease writing experience, \texttt{onchar=fonts id} option can be used when declaring \texttt{\textbackslash babelprovide} to automate the language environment switching.
Note that for some reason, we also need \texttt{main} option when declaring Thai language; otherwise, it will mess with the spacing when writing a Thai paragraph.

Here is some random equation:
\begin{equation}
	f^{(n)}(z_0)=\frac{n!}{2\pi i}\oint_C \frac{f(z)}{(z-z_0)^{n+1}}\,dz.
\end{equation}

Test \textsf{test ทดสอบ} ทดสอบ

\subsection{ภาษาไทย}
\begin{onehalfspace}
	ในย่อหน้าที่\textbf{ภาษาไทย}เป็นหลักควรตั้งระยะห่างระหว่างบรรทัดด้วยคำสั่ง \texttt{\textbackslash onehalfspacing} หรือเรียก environment ชื่อ \texttt{onehalfspace}
\end{onehalfspace}

\begin{onehalfspace}
	คอรัปชันจุ๊ยโปรดิวเซอร์ สถาปัตย์จ๊าบ แจ็กพ็อต ม้าหินอ่อน ซากุระคันถธุระ ฟีดสตาร์ท งี้ บอยคอตอิ่มแปร้สังโฆคำสาปแฟนซี ศิลปวัฒนธรรมไฟลท์จิ๊กโก๋กับดัก $(x+y)^2$ เจลพล็อตมาม่าซากุระดีลเลอร์ ซีนดัมพ์ แฮปปี้ เอ๊าะอุรังคธาตุซิม ฟินิกซ์เทรลเล่อร์อวอร์ด แคนยอนสมาพันธ์ ครัวซองฮัมอาข่าเอ็กซ์เพรส

	คอรัปชันจุ๊ยโปรดิวเซอร์ สถาปัตย์จ๊าบ \textit{แจ็กพ็อต} \textbf{ม้าหินอ่อน} abcd ccddeeff \texttt{ซากุระคันถธุระ} ฟีดสตาร์ท งี้ บอยคอตอิ่มแปร้สังโฆคำสาปแฟนซี 哈羅~ศิลปวัฒนธรรม \textsf{ไฟลท์จิ๊กโก๋กับดัก} เจลพล็อตมาม่าซากุระดีลเลอร์ ซีนดัมพ์ แฮปปี้ เอ๊าะอุรังคธาตุซิม ฟินิกซ์เทรลเล่อร์อวอร์ด แคนยอนสมาพันธ์ ครัวซองฮัมอาข่าเอ็กซ์เพรส
\end{onehalfspace}


\section{CJK scripts using \texttt{xeCJK}}
Here, we use \texttt{UKai TW} and \texttt{Source Han} fonts.
\subsection{繁體字}
兩個黃鸝\textsf{鳴翠柳,} \textbf{一行白鷺上青天。}
\texttt{窗含西嶺千秋雪,}\textit{東吳萬里船。}

{%\CJKfontspec{Noto Serif CJK JP}
	\subsection{日本語}
	秋来ぬと、
	目にはさやかに\texttt{見えねども}、
	\textbf{風の音にぞおどろかれぬる。}
	目にはさやかに\textsf{見えねども}、
	\textit{風の音にぞおどろかれぬる。}
}

% \subsection{韓語}
% 오늘이오늘이소서매일이오늘이소서
% 저물지도새지도말으시고
% 새려면늘언제나오늘이소서

\end{document}
