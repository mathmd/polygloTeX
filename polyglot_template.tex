% !TEX program = xelatex
\documentclass{article}

%% ---- Set up paper margin ---- %%
%\usepackage[a4paper,top=1in,bottom=1in,left=1in,right=1in]{geometry}

%% ---- allow CJK usage ---- %%
\usepackage[CJKspace]{xeCJK} % this should be called before Polyglossia
\setCJKmainfont{Noto Serif CJK TC}
\setCJKsansfont{Noto Sans CJK TC}
\setCJKmonofont{Noto Sans Mono CJK TC}
% \setCJKmainfont{Noto Serif JP}
% \setCJKsansfont{Noto Sans JP}
% \setCJKmonofont{Noto Sans Mono CJK JP}

%% ---- ตั้งค่าให้ตัดคำภาษาไทย ---- %%
\XeTeXlinebreaklocale "th"
\XeTeXlinebreakskip = 0pt plus 0pt % เพิ่มความกว้างเว้นวรรคให้ความยาวแต่ละบรรทัดเท่ากัน

%% ---- font settings ---- %%
\usepackage{fontspec}
\defaultfontfeatures{Mapping=tex-text} % map LaTeX formating, e.g., ``'', to match the current font
% To change the main font, uncomment one of the below command.
% \setmainfont{TeX Gyre Termes} % Free Times
% \setsansfont{TeX Gyre Heros} % Free Helvetica
% \setmonofont{TeX Gyre Cursor} % Free Courier
\newfontfamily{\thaifont}[Scale=MatchUppercase,Mapping=textext]{Laksaman} % ตั้งฟอนต์หลักภาษาไทย
\newenvironment{thailang}{\thaifont}{} % create environment for Thai language
\usepackage[Latin,Thai]{ucharclasses} % ตั้งค่าให้ใช้ "thailang" environment เฉพาะ string ที่เป็น Unicode ภาษาไทย
\setTransitionTo{Thai}{\begin{thailang}}
\setTransitionFrom{Thai}{\end{thailang}}

%% ---- spacing between lines ---- %%
\usepackage{setspace}
% \singlespacing % default setting
% \onehalfspacing % recommend using this for Thai language

%% ---- using alphabatic language ---- %%
\usepackage{polyglossia}
\setdefaultlanguage{english} % it is preferrable to set English as the main language, since the numeric system is compatible with most LaTeX features such as 'enumerate' and so on
\setotherlanguages{thai}

\AtBeginDocument\captionsthai % allow captions to be in Thai

%% ---- math packages ---- %%
\usepackage{amsmath}
\usepackage{amssymb}
\usepackage{bm} % same functionality as \mathbf{} but for greek letters
\numberwithin{equation}{section} % equation numbers are formatted as <#Section>.<#eq in the section>

%% ---- define math environment ---- %%
\usepackage{amsthm}
\newtheorem{definition}{Definition}[section]
\newtheorem{proposition}[definition]{Proposition}
\newtheorem{theorem}[definition]{Theorem}
\newtheorem{corollary}{Corollary}[definition]
\newtheorem{remark}{Remark}[definition]
\newtheorem{lemma}[definition]{Lemma}

%% ---- hyperref settings ---- %%
\usepackage{hyperref}
\usepackage{url}
\usepackage{cite}
\usepackage{xcolor}
\hypersetup{
    colorlinks,
    linkcolor={red!50!black},
    citecolor={blue!50!black},
    urlcolor={blue!80!black}
    }

%% ---- misc. packages ---- %%
\usepackage{enumitem} % allow customizing list environments: enumerate, itemize and description.
\usepackage{mhchem} % use chemistry notation
\usepackage{metalogo} % for extended LaTeX logo such as XeTeX

%% ---- title, authors, and dates ---- %%
\usepackage{authblk}
\title{Multilingual typesetting \XeTeX ~document template}
\author[1]{Chanoknun Sintavanuruk \\ (\textthai{ชนกนันท์ สินธวานุรักษ์}; 馬予棟)}
% \affil[1]{Department of Physiology, Faculty of Medicine Siriraj Hospital, Mahidol University}
% \author[2,3]{XXXX XXXX}
% \affil[2]{Department of XXXX, XXXX University}
% \affil[3]{Department of XXXX, XXXX University}
\date{\today}

\begin{document}
\sloppy % ช่วยตัดคำภาษาไทย
\maketitle
\begin{abstract}
	This document use \texttt{polyglossia} and \texttt{xeCJK} packages to provide multilingual typesetting in Thai, Chinese, Japanese and Korean. 
	ฟอนต์ภาษาไทย \textenglish{\texttt{Laksaman}} เป็นของโครงการ \textenglish{TLWG (Thai Linux Working Group)} ที่มีความคล้ายกับ \textenglish{\texttt{TH SarabunPSK}} ที่ใช้สำหรับการจัดทำเอกสารใน MS Word	
\end{abstract}

\section{Alphabatic languages}
Alphabatic scripts, e.g., Thai, Devanagari, Arabic, Cyrillic, etc., can be used simultaneously by using \texttt{polyglossia} package.
To ease writing experience, \texttt{ucharclasses} package can be used to automate the language environment switching while the main language remains English.

Here is some random equation:
\begin{equation}
	f^{(n)}(z_0)=\frac{n!}{2\pi i}\oint_C \frac{f(z)}{(z-z_0)^{n+1}}\,dz.
\end{equation}

\subsection{ภาษาไทย}
\begin{onehalfspace}
	เนื่องจากว่าใช้แพ็คเกจ \texttt{ucharclasses} เราจึงไม่จำเป็นต้องเรียก environment ภาษาไทยทุกครั้ง ทำให้สามารถใช้หลายภาษาพร้อมกันได้โดยสะดวก
	ในย่อหน้าที่\textbf{ภาษาไทย}เป็นหลักควรตั้งระยะห่างระหว่างบรรทัดด้วยคำสั่ง \texttt{\textbackslash onehalfspacing} หรือเรียก environment ชื่อ \texttt{onehalfspace}
\end{onehalfspace}

\section{CJK languages using \texttt{xeCJK}}
Here, we use \texttt{noto sans CJK} and \texttt{noto serif CJK} fonts.
\subsection{繁體字}
兩個黃鸝鳴翠柳,一行白鷺上青天。
\texttt{窗含西嶺千秋雪,門泊東吳萬里船。}

{\CJKfontspec{Noto Serif CJK JP}
\subsection{日本語}
秋来ぬと、
目にはさやかに見えねども、
\textbf{風の音にぞおどろかれぬる。}
}
\end{document}