% !TEX program = xelatex
\documentclass{article}
%% ---- Set up paper margin ---- %%
%\usepackage[a4paper,top=1in,bottom=1in,left=1in,right=1in]{geometry}

%% ---- Headers ---- %%
\usepackage{xeCJK}
\usepackage{fontspec}
\usepackage{polyglossia}
\setdefaultlanguage{english}
\setotherlanguages{thai}
\newfontfamily\thaifont{Laksaman:script=thai}
% ตั้งค่าฟอนต์แบบความกว้างคงที่ (monospace) ของเอกสาร
%\newfontfamily\thaifonttt{Tlwg Typist:script=thai}
% ตั้งค่าให้ตัดคำภาษาไทย
\XeTeXlinebreaklocale "th"

\usepackage{amsmath}
\usepackage{amsthm}
\usepackage{amssymb}

\usepackage{hyperref}

\usepackage{metalogo}
\title{Multilingual typesetting \XeTeX ~document template}
\author{Chanoknun Sintavanuruk \\ (\textthai{ชนกนันท์ สินธวานุรักษ์}; 馬予棟)}

\begin{document}
	\maketitle
	Hello world. 素晴らしきこの世界。 \textthai{สวัสดีเจ้าโลก}
	
	\begin{abstract}
		This document use \texttt{polyglossia} and \texttt{xeCJK} packages to provide multilingual typesetting in Thai, Chinese, Japanese and Korean. 
		\begin{thai}
			ฟอนต์ภาษาไทย \textenglish{\texttt{Laksaman}} เป็นของโครงการ \textenglish{TLWG (Thai Linux Working Group)} 
			ที่มีความคล้ายกับ \textenglish{\texttt{TH SarabunPSK}} ที่ใช้สำหรับการจัดทำเอกสารใน MS Word	
		\end{thai}
	\end{abstract}
	
	\section{華語、日本語、\textthai{ภาษาไทย}}

	something something

	\begin{equation}
		f^{(n)}(z_0)=\frac{n!}{2\pi i}\oint_C \frac{f(z)}{(z-z_0)^{n+1}}\,dz.
	\end{equation}
	
\end{document}