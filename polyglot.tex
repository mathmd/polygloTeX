%% ---- ตั้งค่าให้ตัดคำภาษาไทย ---- %%
\XeTeXlinebreaklocale "th"
\XeTeXlinebreakskip = 0pt plus 0pt % เพิ่มความกว้างเว้นวรรคให้ความยาวแต่ละบรรทัดเท่ากัน

%% ---- font settings ---- %%
\usepackage{fontspec}
% \usepackage{xltxtra}
% \usepackage{fonts-tlwg}
\defaultfontfeatures{Mapping=tex-text} % map LaTeX formating, e.g., ``'', to match the current font
% To change the main font, uncomment one of the below command.
% \setmainfont{TeX Gyre Termes} % Free Times
% \setsansfont{TeX Gyre Heros} % Free Helvetica
% \setmonofont{TeX Gyre Cursor} % Free Courier
\newfontfamily{\thaifont}[%
    % Scale=MatchUppercase,
    % Mapping=textext,
    ItalicFont={Laksaman-Italic.otf},%
    BoldFont={Laksaman-Bold.otf},%
    BoldItalicFont={Laksaman-BoldItalic.otf},%
    Script=Thai,%
    Scale=MatchLowercase,%
    WordSpace=1.25,%
    Mapping=tex-text,%
]{Laksaman.otf} % ตั้งฟอนต์หลักภาษาไทย
\newenvironment{thailang}{\thaifont}{} % create environment for Thai language
\usepackage[Latin,Thai]{ucharclasses} % ตั้งค่าให้ใช้ "thailang" environment เฉพาะ string ที่เป็น Unicode ภาษาไทย
\setTransitionTo{Thai}{\begin{thailang}}
\setTransitionFrom{Thai}{\end{thailang}}
\newfontfamily\thaifontsf{IBM Plex Sans Thai}
\newfontfamily\thaifonttt{TlwgTypist.otf}
\newcommand{\thaisf}[1]{{\uccoff\thaifontsf #1\uccon}}
\newcommand{\thaitt}[1]{{\uccoff\thaifonttt #1\uccon}}

%% ---- allow CJK usage ---- %%
\usepackage{xeCJK} % this should be called before Polyglossia
% \setCJKmainfont[
%     Path = {\string~/.local/share/fonts/},
%     Extension = .ttf,
%     % AutoFakeBold=true,
%     % AutoFakeSlant=true
%     ]{NotoSerifCJKtc-VF}
% \setCJKsansfont[
%     Path = {\string/usr/share/fonts/google-noto-sans-cjk-vf-fonts/},
%     Extension = .ttf
%     ]{NotoSansCJK-VF}
% \setCJKmonofont[
%     Path = {\string/usr/share/fonts/google-noto-sans-mono-cjk-vf-fonts/},
%     Extension = .ttf
%     ]{NotoSansMonoCJK-VF}
% \setCJKmainfont{Source Han Serif JP}
\setCJKmainfont[ ItalicFont=AR PL UKai TW ]{Source Han Serif JP}
\setCJKsansfont{Source Han Sans JP}
\setCJKmonofont{Source Han Mono}
%%% Define fonts for Japanese and Korean
% \newCJKfontfamily\japanesefont{Noto Serif CJK JP}
% \setCJKmainfont{Noto Serif CJK TC}
% \setCJKsansfont{Noto Sans CJK TC}
% \setCJKmonofont{Noto Sans Mono CJK TC}
% \setCJKmainfont{Noto Serif CJK JP}
% \setCJKsansfont{Noto Sans CJK JP}
% \setCJKmonofont{Noto Sans Mono CJK JP}
% \setCJKmainfont{IPAMincho}
% \setCJKsansfont{IPAGothic}
% \setCJKmonofont{IPAGothic}

%% ---- spacing between lines ---- %%
\usepackage{setspace}
% \singlespacing % default setting
% \onehalfspacing % recommend using this for Thai language

%% ---- using alphabatic language ---- %%
\usepackage{polyglossia}
\setdefaultlanguage{english} % it is preferrable to set English as the main language, since the numeric system is compatible with most LaTeX features such as 'enumerate' and so on
\setotherlanguages{thai}

\AtBeginDocument\captionsthai % allow captions to be in Thai
